\documentclass[14pt]{beamer}
\title{BVRIT HYDERABAD College of Engineering for Women}
\subtitle{FUTOSHIKI - Team 17}

\usefonttheme{serif}
\usepackage{bookman}
\usepackage{hyperref}
\usepackage[T1]{fontenc}
\usepackage{graphicx}
\usecolortheme{beaver}
\beamertemplateballitem

\begin{document}
 \begin{frame}
        \titlepage
    \end{frame}
\begin{frame}
 \begin{itemize}
       \item C Srinika Sharma : 20WH1A6651 : CSE - AIML
       \item G Sathwika : 20WH1A0519 : CSE
       \item M Sannitya : 20WH1A1213 : IT
       \item P Harsha Sri : 20WH1A0491 : ECE
       \item G Preethika : 20WH1A0444 : ECE
   \end{itemize}
\end{frame}
   \begin{frame}
	\frametitle{Introduction}
        \begin{figure}
            \includegraphics[scale=0.5]{grid.png}
        \end{figure}
        \begin{itemize}
	    \item Futoshiki is a puzzle game.
	    \item  The objective of the game is to place the numbers such that each row and column contains only one of each digit.
	\end{itemize}
   \end{frame}
   \begin{frame}
	\frametitle{Rules}
         \begin{enumerate}
	    \item In a N x N puzzle the range of number's must be between 1 and N.  
	    \item No digit is repeated within row or column. No duplicates. 
	    \item Inequality constraints must be honoured: 5 > 4, 5 > 3, or 1 < 4.
	\end{enumerate}
  \end{frame}
\begin{frame}
   \frametitle{Approach}
   \begin{itemize}
       \item Taking the input of the numbers from the user
       \item Validity checking 
       \item Solving the puzzle 
       \item Printing the output
   \end{itemize}
\begin{figure}
            \includegraphics[scale=0.5,width=3cm,height=3cm]{grid.png}
	  \caption{Grid}
        \end{figure}
\end{frame}
 \begin{frame}
  \frametitle{Learnings}
  \begin{itemize}
      \item Understood the purpose of Enumerate Function
      \item Starting small and then integrating all the parts
      \item Team work and coordination
      \item To always keep learning 
   \end{itemize}
\end{frame}
\begin{frame}
  \frametitle{Challenges}
    \begin{itemize}
        \item Placing the constraints in between the cells.
        \item Satisfying inequality constraints while solving puzzle.
        \item Printing the output.
    \end{itemize}
\end{frame}
\begin{frame}
	\frametitle{GIT Repo}
	 \begin{figure}
       	\includegraphics[scale=0.5,width=10cm,height=7cm]{git.png}
	\caption{GIT Repo}
   \end{figure}
    \end{frame}
\begin{frame}
  \frametitle{Statistics}
  \begin{itemize}
      \item Number of lines = 86
      \item Number of functions = 4
   \end{itemize}
\end{frame}
\begin{frame}
	\frametitle{Demo}
	 \begin{figure}
       	\includegraphics[scale=0.5,width=10cm,height=7cm]{output1.png}
   \end{figure}
    \end{frame}

 
\end{document}




