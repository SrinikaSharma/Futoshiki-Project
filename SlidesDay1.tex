\documentclass[14pt]{beamer}
\title{FUTOSHIKI}
\subtitle{Team 17}
\date{\today}
\usefonttheme{serif}
\usepackage{bookman}
\usepackage{hyperref}
\usepackage[T1]{fontenc}
\usepackage{graphicx}
\usecolortheme{beaver}
\beamertemplateballitem

\begin{document}
  
  \begin{frame}
        \titlepage
    \end{frame}
  \begin{frame}
	\begin{enumerate}
		\item C Srinika Sharma : 20WH1A6651 : CSE - AIML
		\item M Sannitya : 20WH1A1213 : IT 
		\item G Sathwika : 20WH1A0519 : CSE
		\item P Harsha Sri : 20WH1A0491 : ECE
		\item G Preethika : 20WH1A0444 : ECE
	\end{enumerate}
  \end{frame}


  \begin{frame}
	\frametitle{Introduction}
        
	\begin{itemize}
	    \item Futoshiki is a puzzle game which originated in Japan and it means inequality. The puzzle is played on a square grid. The objective of the game is to place the numbers such that each row and column contains only one of each digit.
	\end{itemize}
	
  \end{frame}
   \begin{frame}
	\frametitle{Rules}

	\begin{enumerate}
	    \item In a N x N puzzle the range of number's must be between 1 and N.  
	    \item No digit is repeated within row or column. No duplicates. 
	    \item Inequality constraints must be honoured: 5 > 4, 5 > 3, or 1 < 4.
	\end{enumerate}
  \end{frame}
 
 \begin{frame}
 \frametitle{GRID}
 \begin{figure}
 \includegraphics[scale=0.5]{grid.png}
 \end{figure}
 \end{frame}
   
  \begin{frame}
	\frametitle{Day 2 - October 21-10-2021}
        
	\begin{itemize}
	    \item We have worked on the logic to solve the problem statement and have also tried to figure out the approach to the main code. 
	\end{itemize}
	
  \end{frame}


\end{document}
