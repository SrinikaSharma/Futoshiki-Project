\documentclass[14pt]{beamer}
\title{FUTOSHIKI}
\subtitle{Team 17}
\date{\today}
\author[Bvrith]{ C Srinika Sharma : 20WH1A6651 : CSE - AIML\\G Sathwika : 20WH1A0519 : CSE\\M Sannitya : 20WH1A1213 : IT \\P Harsha Sri : 20WH1A0491 : ECE\\G Preethika : 20WH1A0444 : ECE}
\usefonttheme{serif}
\usepackage{bookman}
\usepackage{hyperref}
\usepackage[T1]{fontenc}
\usepackage{graphicx}
\usecolortheme{beaver}
\beamertemplateballitem

\begin{document}
   \begin{frame}
        \titlepage
    \end{frame}
   \begin{frame}
	\frametitle{Introduction}
        \begin{figure}
            \includegraphics[scale=0.5]{grid.png}
        \end{figure}
        \begin{itemize}
	    \item Futoshiki is a puzzle game which originated in Japan and it means inequality.It is played on a square grid. The objective of the game is to place the numbers such that each row and column contains only one of each digit.
	\end{itemize}
   \end{frame}
   \begin{frame}
	\frametitle{Rules}
         \begin{enumerate}
	    \item In a N x N puzzle the range of number's must be between 1 and N.  
	    \item No digit is repeated within row or column. No duplicates. 
	    \item Inequality constraints must be honoured: 5 > 4, 5 > 3, or 1 < 4.
	\end{enumerate}
  \end{frame}
\begin{frame}
   \frametitle{Approach}
   \begin{itemize}
       \item Taking the input of the numbers from the user
       \item Validity checking - No duplicates in respective rows and columns and constraint checking.
       \item Solving the puzzle by assigning the number to the cell only if all the conditions are satisfied
       \item Printing the output.
   \end{itemize}
\end{frame}
\begin{frame}
     \frametitle{Work progress}
     \begin{itemize}
         \item Day 1:Created a project group in Gitlab,Learned about working with LaTeX,Understood the problem statement. 
         \item Day 2:Worked on the logic to solve the problem statement and have also tried to figure out the approach to the main code.
         \item Day 3:Worked on Code.
         \item Day 4:Worked on placing the constraints between the cells in the code,making a grid.
         \item Day 5:Checking the conditions of constraints and finishing the code.
     \end{itemize}
\end{frame}
 \begin{frame}
  \frametitle{Learnings}
  \begin{itemize}
      \item learned working with LaTeX.
      \item Working with Gitlab.
      \item Working with pygame.
   \end{itemize}
\end{frame}
\begin{frame}
	\frametitle{GIT Repo}
	 \begin{figure}
       	\includegraphics[scale=0.5,width=10cm,height=7cm]{gitrepo.png}
   \end{figure}
    \end{frame}
\begin{frame}
  \frametitle{Challenges}
    \begin{itemize}
        \item Placing the constraints in between the cells.
        \item Satisfying inequality constraints while solving puzzle.
        \item Representing the output in the form of grid.
    \end{itemize}
\end{frame}
\begin{frame}
     \centering \Huge
     \emph{THANK YOU}
\end{frame}
 
\end{document}


