\documentclass[14pt]{beamer}
\title{FUTOSHIKI}
\subtitle{Team 17}
\date{\today}
\author[Bvrith]{ C Srinika Sharma : 20WH1A6651 : CSE - AIML\\G Sathwika : 20WH1A0519 : CSE\\M Sannitya : 20WH1A1213 : IT \\P Harsha Sri : 20WH1A0491 : ECE\\G Preethika : 20WH1A0444 : ECE}
\usefonttheme{serif}
\usepackage{bookman}
\usepackage{hyperref}
\usepackage[T1]{fontenc}
\usepackage{graphicx}
\usecolortheme{orchid}
\beamertemplateballitem

\begin{document}
   \begin{frame}
        \titlepage
    \end{frame}
   \begin{frame}
	\frametitle{Introduction}
        \begin{figure}
            \includegraphics[scale=0.5]{grid.png}
        \end{figure}
        \begin{itemize}
	    \item Futoshiki is a puzzle game which originated in Japan and it means inequality. The puzzle is played on a square grid. The objective of the game is to place the numbers such that each row and column contains only one of each digit.
	\end{itemize}
   \end{frame}
   \begin{frame}
	\frametitle{Rules}
         \begin{enumerate}
	    \item In a N x N puzzle the range of number's must be between 1 and N.  
	    \item No digit is repeated within row or column. No duplicates. 
	    \item Inequality constraints must be honoured: 5 > 4, 5 > 3, or 1 < 4.
	\end{enumerate}
  \end{frame}
\begin{frame}
   \frametitle{Approach}
   \begin{itemize}
       \item Checking possibilities for each cell based on the constraints.
       \item Validity checking - No duplicates in respective rows and columns.
       \item Puzzle is solved like sudoku.
   \end{itemize}
\end{frame}
 \begin{frame}
      \frametitle {Day 1 - October 20-10-2021}
     \begin{itemize}
          \item Created a project group in Gitlab.
          \item Learned about working with LaTeX.
          \item Understood the problem statement.
     \end{itemize}
\end{frame}
   \begin{frame}
	\frametitle{Day 2 - October 21-10-2021}
       \begin{figure}
            \includegraphics[scale=0.5,width=8cm,height=7cm]{activity.jpeg}
       \end{figure}
         \begin{itemize}
	    \item We have worked on the logic to solve the problem statement and have also tried to figure out the approach to the main code. 
	\end{itemize}
  \end{frame}
\begin{frame}
     \frametitle{Day 3 - October 22-10-2021}
    \begin{figure}
        %\includegraphicx{}
   \end{figure}
    \begin{itemize}
       \item Worked on Code
   \end{itemize}
\end{frame}
\begin{frame}
  \frametitle{Learnings}
  \begin{itemize}
      \item learned working with LaTeX.
      \item Working with Gitlab.
   \end{itemize}
\end{frame}
\begin{frame}
  \frametitle{Challenges}
    \begin{itemize}
        \item Placing the constraints in between the cells.
        \item Satisfying inequality constraints while solving puzzle.
        \item Representing the output in the form of grid.
    \end{itemize}
\end{frame}
\begin{frame}
     \centering \Huge
     \emph{THANK YOU}
\end{frame}
 
\end{document}

